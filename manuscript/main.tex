% Use only LaTeX2e, calling the article.cls class and 12-point type.
% modified by Aaron Clauset (2014) from the scifile.tex file distributed
% by AAAS for articles in Science

\documentclass[12pt]{article}

% Users of the {thebibliography} environment or BibTeX should use the
% scicite.sty package, downloadable from *Science* at
% www.sciencemag.org/about/authors/prep/TeX_help/ .
% This package should properly format in-text
% reference calls and reference-list numbers.

\usepackage{scicite}

% Use times if you have the font installed; otherwise, comment out the
% following line.

\usepackage{times}

% Some standard mathematical notation and figure packages

\usepackage{amsmath}
\usepackage{amsfonts}
\usepackage{amssymb}
\usepackage{graphicx}

% The preamble here sets up a lot of new/revised commands and
% environments.  It's annoying, but please do *not* try to strip these
% out into a separate .sty file (which could lead to the loss of some
% information when we convert the file to other formats).  Instead, keep
% them in the preamble of your main LaTeX source file.


% The following parameters seem to provide a reasonable page setup.

\topmargin 0.0cm
\oddsidemargin 0.2cm
\textwidth 16cm 
\textheight 21cm
\footskip 1.0cm


%The next command sets up an environment for the abstract to your paper.

\newenvironment{sciabstract}{%
\begin{quote} \bf}
{\end{quote}}


% If your reference list includes text notes as well as references,
% include the following line; otherwise, leave it commented out. 

%\renewcommand\refname{References and Notes}


% Include your paper's title here

\title{Individual Risk Attitudes and Risk Perception Make a Difference for the Acceptance of Contact Tracing} 


% Place the author information here.  Please hand-code the contact
% information and notecalls; do *not* use \footnote commands.  Let the
% author contact information appear immediately below the author names
% as shown.  We would also prefer that you don't change the type-size
% settings shown here.

% Authors should be listed in order of contribution to the paper beneath the title on the opening page of the manuscript. Use first name, then middle initial (if any), followed by last name with each name separated by commas. The author list should be one single paragraph with no line breaks.

\author
{Rebecca Albrecht,${}^{1\ast}$ Jana B. Jarecki,${}^{1}$ Dominik Meier${}^{1}$, Jörg Rieskamp${}^{1}$\\
\normalsize{${}^{1}$Center for Economic Psychology}\\
\normalsize{Missionsstrasse 62a, 4055 Basel, Switzerland}\\
\normalsize{$^\ast$rebecca.albrecht@unibas.ch}
}

% Include the date command, but leave its argument blank.

\date{}



%%%%%%%%%%%%%%%%% END OF PREAMBLE %%%%%%%%%%%%%%%%

% Submission guidlines: https://advances.sciencemag.org/content/information-authors
% Text: maximum 15,000 words
% Abstract: maximum 150 words
% May include up to six figures and/or tables and about 40 references


\begin{document} 

% Double-space the manuscript.

\baselineskip24pt

% Make the title.

\maketitle 



% Place your abstract within the special {sciabstract} environment.

% The abstract should be a single paragraph, not to exceed 250 words and ideally closer to 200, written in plain language that a general reader can understand. It should include
% An opening sentence that states the question/problem addressed by the research AND
% Enough background content to give context to the study AND
% A brief statement of primary results AND
% A short concluding sentence.
% Do not include citations or undefined abbreviations in the abstract. Any abbreviations that appear in the title should be defined in the abstract.

\begin{sciabstract}
  Abstract to be Done
\end{sciabstract}



% In setting up this template for *Science Advances* papers, both
% the \section* command and the \paragraph* command are used for topical
% divisions.  Which you use will of course depend on the type of paper
% you're writing.  Review Articles tend to have displayed headings, for
% which \section* is more appropriate; Research Articles, when they have
% formal topical divisions at all, tend to signal them with bold text
% that runs into the paragraph, for which \paragraph* is the right
% choice.  Either way, use the asterisk (*) modifier, as shown, to
% suppress numbering.

\paragraph*{Introduction}
% The manuscript should start with a brief introduction that lays out the problem addressed by the research and describes the paperÕs importance. The scientific question being investigated should be described in detail. The introduction should provide sufficient background information to make the article understandable to readers in other disciplines, and provide enough context to ensure that the implications of the experimental findings are clear.

% In this file, we present some tips and sample mark-up to assure your
% \LaTeX\ file of the smoothest possible journey from review manuscript
% to published {\it Science Advances\/} paper.  We focus here particularly on
% issues related to style files, citation, and math, tables, and
% figures, as those tend to be the biggest sticking points.  Please use
% the source file for this document, \texttt{sciadvfile.tex}, as a template
% for your manuscript, cutting and pasting your content into the file at
% the appropriate places.

% {\it Science Advances\/}'s publication workflow relies on Microsoft Word.  To
% translate \LaTeX\ files into Word, AAAS uses an intermediate MS-DOS
% routine \cite{tth} that converts the \TeX\ source into HTML\@.  The
% routine is generally robust, but it works best if the source document
% is clean \LaTeX\ without a significant freight of local macros or
% \texttt{.sty} files.  Use of the source file \texttt{sciadvfile.tex} as a
% template, and calling {\it only\/} the \texttt{.sty} and \texttt{.bst}
% files specifically mentioned here, will generate a manuscript that
% should be eminently reviewable, and yet will allow your paper to
% proceed quickly into our production flow upon acceptance \cite{use2e}.



\section*{Results}
% The results should describe the experiments performed and the findings observed. The results section should be divided into subsections to delineate different experimental themes. Subheadings should either be all phrases or all complete sentences. All data must be shown either in the main text or in the Supplementary Materials.
% 
% All data should be presented in the Results. No data should be presented for the first time in the Discussion. Data (such as from Western blots) should be appropriately quantified.
% Subheadings must be either all complete sentences or all phrases. They should be brief, ideally less than 10 words. Subheadings should not end in a period. Your paper may have as many subheadings as are necessary.
% Figures and tables must be called out in numerical order. For example, the first mention of any panel of Fig. 3 cannot precede the first mention of all panels of Fig. 2. The supplementary figures (for example, fig. S1) and tables (table S1) must also be called out in numerical order.
% Display equations (set on their own line) can be included. Do not use the native Word 2007, 2008, 2010, or 2011 equation editor. This can in produce inaccurate MathML, the online markup language we use, which may result in display errors. Instead, use the legacy equation editor in Word (Insert menu; select insert object; select word equation) or use MathType (recommended). If you enter equations in simple LaTeX, check that they will convert accurately (Word 2007 and higher can convert simple LaTeX equations). Display equations should be numbered at the rightÑ(1), (2), etc.
% The same guidelines apply to mathematical expressions within a sentence of text; however, MathType (or the equivalent) should be used within text only when the desired result cannot be achieved using ordinary Word characters. Reserve MathType for when its use is unavoidableÑfor example, characters with overbars or carets, with stacked superscripts and subscripts, or within square root symbols.
% All data must be shown; references to Òunpublished resultsÓ or Òdata not shownÓ are not permitted.



% \subsection*{Formatting Citations}

% Citations can be handled in one of three ways.  The most
% straightforward (albeit labor-intensive) would be to hardwire your
% citations into your \LaTeX\ source, as you would if you were using an
% ordinary word processor.  Thus, your code might look something like
% this:


% \begin{quote}
% \begin{verbatim}
% However, this record of the solar nebula may have been
% partly erased by the complex history of the meteorite
% parent bodies, which includes collision-induced shock,
% thermal metamorphism, and aqueous alteration
% ({\it 1, 2, 5--7\/}).
% \end{verbatim}
% \end{quote}


% \noindent Compiled, the last two lines of the code above, of course, would give notecalls in {\it Science Advances\/} style:

% \begin{quote}
% \ldots thermal metamorphism, and aqueous alteration ({\it 1, 2, 5--7\/}).
% \end{quote}

% Under the same logic, the author could set up his or her reference list as a simple enumeration,

% \begin{quote}
% \begin{verbatim}
% {\bf References}

% \begin{enumerate}
% \item G. Gamow, {\it The Constitution of Atomic Nuclei
% and Radioactivity\/} (Oxford Univ. Press, New York, 1931).
% \item W. Heisenberg and W. Pauli, {\it Zeitschr.\ f.\ 
% Physik\/} {\bf 56}, 1 (1929).
% \end{enumerate}
% \end{verbatim}
% \end{quote}

% \noindent yielding

% \begin{quote}
% {\bf References}

% \begin{enumerate}
% \item G. Gamow, {\it The Constitution of Atomic Nuclei and
% Radioactivity\/} (Oxford Univ. Press, New York, 1931).
% \item W. Heisenberg and W. Pauli, {\it Zeitschr.\ f.\ Physik} {\bf 56},
% 1 (1929).
% \end{enumerate}
% \end{quote}

% That's not a solution that's likely to appeal to everyone, however ---
% especially not to users of B{\small{IB}}\TeX\ \cite{inclme}.  If you
% are a B{\small{IB}}\TeX\ user, we suggest that you use the
% \texttt{ScienceAdvances.bst} bibliography style file and the
% \texttt{scicite.sty} package, the latter of which is downloadable from the \textit{Science} author help site
% (\verb+http://www.sciencemag.org/about/authors/prep/TeX_help/+).  You can also
% generate your reference lists by using the list environment
% \texttt{\{thebibliography\}} at the end of your source document; here
% again, you may find the \texttt{scicite.sty} file useful.

% Whether you use B{\small{IB}}\TeX\ or \texttt{\{thebibliography\}}, be
% very careful about how you set up your in-text reference calls and
% notecalls.  In particular, observe the following requirements:

% \begin{enumerate}
% \item Please follow the style for references outlined at our author
%   help site and embodied in recent issues of {\it Science Advances}.  Each
%   citation number should refer to a single reference; please do not
%   concatenate several references under a single number.
% \item Please cite your references and notes in text {\it only\/} using
%   the standard \LaTeX\ \verb+\cite+ command, not another command
%   driven by outside macros.
% \item Please separate multiple citations within a single \verb+\cite+
%   command using commas only; there should be {\it no space\/}
%   between reference keynames.  That is, if you are citing two
%   papers whose bibliography keys are \texttt{keyname1} and
%   \texttt{keyname2}, the in-text cite should read
%   \verb+\cite{keyname1,keyname2}+, {\it not\/}
%   \verb+\cite{keyname1, keyname2}+.
% \end{enumerate}

% \noindent Failure to follow these guidelines could lead
% to the omission of the references in an accepted paper when the source
% file is translated to Word via HTML.

% \subsection*{Handling Math, Tables, and Figures}

% Following are a few things to keep in mind in coding equations,
% tables, and figures for submission to {\it Science Advances}.

% \paragraph*{In-line math.}  The utility that AAAS uses for converting
% from \LaTeX\ to HTML handles in-line math relatively well.  It is best
% to avoid using built-up fractions in in-line equations, and going for
% the more boring ``slash'' presentation whenever possible --- that is,
% for \verb+$a/b$+ (which comes out as $a/b$) rather than
% \verb+$\frac{a}{b}$+ (which compiles as $\frac{a}{b}$).  Likewise,
% HTML isn't tooled to handle certain overaccented special characters
% in-line; for $\hat{\alpha}$ (coded \verb+$\hat{\alpha}$+), for
% example, the HTML translation code will return [\^{}$(\alpha)$].
% Don't drive yourself crazy --- but if it's possible to avoid such
% constructs, please do so.  Please do not code arrays or matrices as
% in-line math; display them instead.  And please keep your coding as
% \TeX-y as possible --- avoid using specialized math macro packages
% like \texttt{amstex.sty}.

% \paragraph*{Displayed math.} The AAAS HTML converter sets up \TeX\
% displayed equations using nested HTML tables.  That works well for an
% HTML presentation, but Word chokes when it comes across a nested
% table in an HTML file.  That problem is circumvented by simply cutting the
% displayed equations out of the HTML before it's imported into Word,
% and then replacing them in the Word document using either images or
% equations generated by a Word equation editor.  Strictly speaking,
% this procedure doesn't bear on how you should prepare your manuscript
% --- although, for reasons best consigned to a note \cite{nattex}, AAAS would
% prefer that you use native \TeX\ commands within displayed-math
% environments, rather than \LaTeX\ sub-environments.

% \paragraph*{Tables.}  The HTML converter that AAAS uses seems to handle
% reasonably well simple tables generated using the \LaTeX\
% \texttt{\{tabular\}} environment.  For very complicated tables, you
% may want to consider generating them in a word processing program and
% including them as a separate file.

% \paragraph*{Figures.}  Figure callouts within the text should not be
% in the form of \LaTeX\ references, but should simply be typed in ---
% that is, \verb+(Fig. 1)+ rather than \verb+\ref{fig1}+.  For the
% figures themselves, treatment can differ depending on whether the
% manuscript is an initial submission or a final revision for acceptance
% and publication.  For an initial submission and review copy, you can
% use the \LaTeX\ \verb+{figure}+ environment and the
% \verb+\includegraphics+ command to include your PostScript figures at
% the end of the compiled PostScript file.  For the final revision,
% however, the \verb+{figure}+ environment should {\it not\/} be used;
% instead, the figure captions themselves should be typed in as regular
% text at the end of the source file (an example is included here), and
% the figures should be uploaded separately according to the Art
% Department's instructions.




\section*{Discussion}
% The discussion describes the conclusions that can be drawn from the results, as well as the significance and implications of the research. A paragraph discussing the limitations of the study should be included and any issues that will need to be addressed before application to animal, human, or environmental health should also be described.

% \subsection*{What to Send In}

% What you should send to {\it Science Advances\/} will depend on the stage your manuscript is in:

% \begin{itemize}
% \item {\bf Important:} If you're sending in the initial submission of
%   your manuscript (that is, the copy for evaluation and peer review),
%   please send in {\it only\/} a PostScript or PDF version of the
%   compiled file (including figures).  Please do not send in the \TeX\ 
%   source, \texttt{.sty}, \texttt{.bbl}, or other associated files with
%   your initial submission.  (For more information, please see the
%   instructions at the {\it Science Advances\/} web site,
%   http://www.scienceadvances.org/ .)
% \item When the time comes for you to send in your revised final
%   manuscript (i.e., after peer review), AAAS requires that you include
%   all source files and generated files in your upload.  Thus, if the
%   name of your main source document is \texttt{ltxfile.tex}, you
%   need to include:
% \begin{itemize}
% \item \texttt{ltxfile.tex}.
% \item \texttt{ltxfile.aux}, the auxilliary file generated by the
%   compilation.
% \item A PostScript file (compiled using \texttt{dvips} or some other
%   driver) of the \texttt{.dvi} file generated from
%   \texttt{ltxfile.tex}, or a PDF file distilled from that
%   PostScript.  You do not need to include the actual \texttt{.dvi}
%   file in your upload.
% \item From B{\small{IB}}\TeX\ users, your bibliography (\texttt{.bib})
%   file, {\it and\/} the generated file \texttt{ltxfile.bbl} created
%   when you run B{\small{IB}}\TeX.
% \item Any additional \texttt{.sty} and \texttt{.bst} files called by
%   the source code (though, for reasons noted earlier, we {\it
%     strongly\/} discourage the use of such files beyond those
%   mentioned in this document).
% \end{itemize}
% \end{itemize}


% \noindent \textbf{Supplementary Material} accompanies this paper at {\small {\tt http://www.scienceadvances.org/}}.



\section*{Materials and Methods}
% The materials and methods section should provide sufficient information to allow replication of the results. Begin with a section titled Experimental Design describing the objectives and design of the study as well as pre-specified components.
% 
% In addition, include a section titled Statistical Analysis at the end that fully describes the statistical methods with enough detail to enable a knowledgeable reader with access to the original data to verify the results. The values for N, P, and the specific statistical test performed for each experiment should be included in the appropriate figure legend or main text.
\subsection*{Preregistered Hypothesis}
The main hypotheses about the influence of risk perception, risk preferences and social values on attitudes towards digital contact tracing apps (DCT) were preregistered (https://osf.io/b3ud5). Our first set of hypotheses concerned the subjective \textit{perception of risks}: (H1a) Higher risk perception regarding health risks through COVID-19 increases the acceptance of and compliance with DCT. (H1b) Higher risk perception regarding data security reduces the acceptance of and compliance with DCT. (H1c) Higher risk perception regarding economic risks through COVID-19 increases the acceptance of and compliance with DCT.

Regarding the personal \textit{preferences for seeking risks}, we hypothesized that (H2a) seeking more risks in general decreases the acceptance of and compliance with DCT. (H2b) Seeking health risks decreases the acceptance of and compliance with DCT. (H2c) Seeking risks regarding data security, however, increases the acceptance of and compliance with DCT. (H2d). Seeking economic risks decreases the acceptance of and compliance with DCT.

Regarding \textit{social preferences and personality} we hypothesized that (H3a) the higher general other-regarding social preferences the higher the acceptance of and compliance with DCT. (H3b) The higher the social preference towards the local versus the global community the higher the acceptance and compliance with tracing apps. (H3c) The higher the general honesty-humility personality trait the higher the acceptance of and compliance with DCT.


\subsection*{Data}
The data source is a representative sample of the Swiss German-speaking population, who took part in an online survey. In total 848 participants recruited through a professional panel provider completed the survey, 91 had to excluded for insincere responding\footnote{Incorrect answers to several explicit attention check items or self-reported lack of data quality}, leaving a final sample of \textit{N} = 757; 388 men, 366 women, 3 refused to report gender (51.3\%, 48.3\% and 0.4\%, respectively), the mean age was 45 years (\textit{Mdn} = 44, \textit{SD} = 16, range 18-79 years), data were collected in July 2020, the study was approved by the ethics committee of the faculty of Psychology at the University of Basel.


\subsection*{Statistical Analyses}
Following our preregistered methodology we tested the hypotheses using a Bayesian linear regression, which included the main variables as predictors and included those covariates that yield the most parsimoneous model (see covariate selection).
\paragraph*{Covariate Selection}
The covariates in the regression model were selected using a (Bayesian) projection onto a reference model, which is a variable selection method that has been shown to outperform other method in selecting the variables that balance model sparsity and predictive accuracy \cite{Pavone2020UsingSelection, Piironen2020ProjectiveSelection}. Such variable selection method constructs a reference model (we used the full model) and searches for a reduced model with minimal loss of performance compared to the reference model. The simpler model is constructed by a projection of the model parameters from the full model, because exhausting all possible combinations of predictor variables is infeasible.



% Your references go at the end of the main text, and before the
% figures.  For this document we've used BibTeX, the .bib file
% scibib.bib, and the .bst file Science.bst.  The package scicite.sty
% was included to format the reference numbers according to *Science*
% style.

\bibliography{references}
\bibliographystyle{ScienceAdvances}


\noindent \textbf{Acknowledgements:} 
% Acknowledgments should be gathered into a paragraph after the final numbered reference. This section should also include 
% * complete funding information, 
% * a description of each authorÕs contribution to the paper, 
% * a listing of any competing interests of any of the authors (all authors must also fill out the Conflict of Interest form), and, 
% * a section on data and materials availability, information about the location of the data if not included in the paper, including **accession numbers** to any data relating to the paper and deposited in a public database.
%
% The authors thank James Clerk Maxwell and Albert Einstein for helpful conversations.\\
% \noindent \textbf{Funding:} This work was supported in part by the Very Generous Foundation.\\
% \noindent \textbf{Author Contributions} JAS conceived the research. JAS and JD designed the analyses. JAS and JS conducted the analyses. All authors wrote the manuscript.\\
% \noindent \textbf{Competing Interests} The authors declare that they have no competing financial interests.\\
% \noindent \textbf{Data and materials availability:} Additional data and materials are available online.





% For your review copy (i.e., the file you initially send in for
% evaluation), you can use the {figure} environment and the
% \includegraphics command to stream your figures into the text, placing
% all figures at the end.  For the final, revised manuscript for
% acceptance and production, however, PostScript or other graphics
% should not be streamed into your compliled file.  Instead, set
% captions as simple paragraphs (with a \noindent tag), setting them
% off from the rest of the text with a \clearpage as shown  below, and
% submit figures as separate files according to the Art Department's
% instructions.


\clearpage

\noindent {\bf Fig. 1.} Please do not use figure environments to set
up your figures in the final (post-peer-review) draft, do not include graphics in your
source code, and do not cite figures in the text using \LaTeX\
\verb+\ref+ commands.  Instead, simply refer to the figure numbers in
the text per {\it Science\/} style, and include the list of captions at
the end of the document, coded as ordinary paragraphs as shown in the
\texttt{sciadvfile.tex} template file.  Your actual figure files should
be submitted separately.



\end{document}




















